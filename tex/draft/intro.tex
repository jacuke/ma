\documentclass[12pt,fleqn]{article}
\usepackage[utf8]{inputenc}
\usepackage[T1]{fontenc}
\usepackage{palatino}
\usepackage{graphicx}
\usepackage{verbatim}
\usepackage[german]{babel}
%\usepackage[a4paper]{geometry}
\usepackage[a4paper, top=2cm, bottom=2cm, left=1.5cm, right=1.5cm]{geometry}
\usepackage{amsmath, amsthm, amssymb, amsfonts}
\usepackage{listings} 
\usepackage{pgfplots}
\usepackage{xcolor}
%\usepackage[dvipsnames]{xcolor}

\usepackage[gen]{eurosym}
\DeclareUnicodeCharacter{20AC}{\euro{}}

%\usepackage[pdfborder={0 0 0}]{hyperref}

%\usepackage{tikz}
%\usetikzlibrary{intersections}

\usepackage{tabularx}
\usepackage{longtable}

\usepackage{parskip}
\setlength{\parindent}{0pt}
\setlength{\mathindent}{0pt}

\usepackage{enumitem}
%\setlist{nolistsep,topsep=0pt,partopsep=0pt}
%\setlist{notopsep}

% für MA: %
\usepackage{multirow}
\usepackage{fancyvrb}
\usepackage{tikz,lipsum,lmodern}
\usepackage[most]{tcolorbox}
\usepackage{float}
\usepackage[german]{struktex}

\usepackage{arydshln}

\pgfplotsset{compat=1.18}

\newcommand{\heightNS}{8}

% nur für Draft-Zitierungen %
\usepackage{natbib}

%\newcommand{\rom}[1]{\uppercase\expandafter{\romannumeral #1\relax}}

\begin{comment}
\usepackage{listings}
\lstset{
basicstyle=\small\ttfamily,
columns=flexible,
breaklines=true
}
\end{comment}

\begin{comment}
\usepackage{fancyvrb}
% redefine \VerbatimInput
\RecustomVerbatimCommand{\VerbatimInput}{VerbatimInput}%
{fontsize=\scriptsize,
 %
 frame=lines,  % top and bottom rule only
 framesep=2em, % separation between frame and text
% rulecolor=\color{Gray},
 %
 label=\fbox{data.txt},
 labelposition=topline,
 %
 commandchars=\|\(\), % escape character and argument delimiters for
                      % commands within the verbatim
 commentchar=*        % comment character
}
\end{comment}

\pagestyle{empty}



\begin{document}

\section{Einleitung}

\subsection{Patienten-Daten}

Medizininformatik-Initiative
\texttt{https://www.medizininformatik-initiative.de/de/start}

Deutsches Forschungsdatenportal für Gesundheit
\texttt{https://forschen-fuer-gesundheit.de}

\section{Normen}

\section{Medicats}

\section{Using Multiple Groups}

\section{SNOMED}

\section{Sources}

FHIR \citep{braunstein2022health}

using multiple groups \citep{saripalle2019representing}

SNOMED \citep{icd10-to-snomed}

\bibliographystyle{IEEEtranSN}
\bibliography{intro}

\begin{comment}
\begingroup
\renewcommand{\section}[2]{}
\begin{thebibliography}{99}

\bibitem{fhir-term}
FHIR Terminology Module \newline
\url{https://build.fhir.org/terminology-module.html}
%https://www.youtube.com/watch?v=Hd8A_gBnpzk

\bibitem{fhir-concept}
FHIR ConceptMap2 \newline
\url{https://hl7.org/fhir/conceptmap.html}

\bibitem{fhir-mule}
Mulesoft FHIR R4 ConceptMap Library \newline
\url{https://anypoint.mulesoft.com/exchange/org.mule.examples/fhir-r4-conceptmap-library/}

\bibitem{medinf-init}
Medizininformatik-Initiative \newline
\url{https://www.medizininformatik-initiative.de/de/start}
% Der Druck andererseits ist groß, mit vorhandenen Versorgungsdaten (insb. in der Medizininformatik-Initiative (LINK), in der auch wir aktiv sind) für die Forschung zu arbeiten.

\bibitem{fdpg}
Deutsches Forschungsdatenportal für Gesundheit \newline
\url{https://forschen-fuer-gesundheit.de}
% Hierzu gibt es eine Forschungsdaten für Gesundheits-Plattform (LINK), wo im großen Stil Daten aus sogenannten Datenintegrationszentren (DIZ) aus etwa 30 Universitätskliniken in Deutschland föderiert abgefragt werden. 

\bibitem{ICD-10-GM}
ICD-10-GM Kodiersystem \newline
\url{https://www.bfarm.de/DE/Kodiersysteme/Klassifikationen/ICD/ICD-10-GM/_node.html}

\bibitem{WHO-FIC-Class}
``WHO-FIC Classifications and Terminology Mapping -- Principles and Best Practice'' \newline
\url{https://cdn.who.int/media/docs/default-source/classification/who-fic-network/whofic_terminology_mapping_guide.pdf?sfvrsn=2cae387c_7&download=true}

\bibitem{ISO/TS 21564:2019}
ISO/TS 21564:2019 --
``Health Informatics, Terminology resource map quality measures (MapQual)'' \newline
\url{https://www.iso.org/standard/71088.html}

\bibitem{ISO/TR 12300:2014}
ISO/TR 12300:2014 --
``Health informatics — Principles of mapping between terminological resources'' \newline
\url{https://www.iso.org/standard/51344.html}

\bibitem{OHDSI Athena}
OHDSI Athena \newline
\url{https://cdn.who.int/media/docs/default-source/classification/who-fic-network/whofic_terminology_mapping_guide.pdf?sfvrsn=2cae387c_7&download=true}
% Ein ganz anderer Ansatz wäre, die Klassifikations-Codes auf Terminologie-Codes (insb. SNOMED CT) zu mappen, um nochmal ganz anders vorzugehen. Ein ähnliches Vorgehen gibt es im OHDSI-Projekt, wo ein interessantes Terminologie-Tool „Athena“ (LINK) verwendet wird.

\bibitem{Medicats}
Medicats Library \newline
\url{https://github.com/hhund/medicats}

\bibitem{Medicats Thesis}
``Medical Classification and Terminology Systems in a Secondary Use Context: Challenges and Perils'' \newline
\url{https://www.researchgate.net/publication/320962066_Medical_Classification_and_Terminology_Systems_in_a_Secondary_Use_Context_Challenges_and_Perils}

\bibitem{ICD-10 to SNOMED CT}
``Evaluation of an Automated Mapping from ICD-10 to SNOMED CT'' \newline
\url{https://american-cse.org/csci2022-ieee/pdfs/CSCI2022-2lPzsUSRQukMlxf8K2x89I/202800b612/202800b612.pdf}
% ODER: Man mappt die Klassen (unterhalb der Klassenebene) auf präzisere SNOMED CT-Codes (das sind mehrere, d.h. welchen nimmt man?) und garantiert so jahres- bzw. versionsübergreifende Auswertungen?

\bibitem{UMLS FHIR}
``An Interoperable UMLS Terminology Service Using FHIR'' \newline
\url{https://www.mdpi.com/1999-5903/12/11/199}
% (Vorgehensweise im Kontext UMLS: das hatten wir ja in der Master-Vorlesung)

\bibitem{UMLS FHIR 2}
``Representing UMLS knowledge using FHIR Terminological Resources'' \newline
\url{https://ieeexplore.ieee.org/abstract/document/8983305}
% (wie vorher zu UMLS; siehe auch Anhang)

\bibitem{OHDSI FHIR}
``Building Interoperable FHIR-Based Vocabulary Mapping Services: A Case Study of OHDSI Vocabularies and Mappings'' \newline
\url{https://ebooks.iospress.nl/doi/10.3233/978-1-61499-830-3-1327}
% (Vorgehensweise im Kontext OHDSI, einem langjährigen Real-World-Evidence-Projekt, wo heterogene Daten auf ein Standard-Datenmodell (OMOP) inkl. Standard-Terminologie (Athena-Tool) gemappt wird.) 

\bibitem{MENDS-on-FHIR}
``MENDS-on-FHIR: Leveraging the OMOP common data model and FHIR standards for national chronic disease surveillance'' \newline
\url{https://www.medrxiv.org/content/10.1101/2023.08.09.23293900v2}
% (wie vorher zum Kontext OHDSI; siehe auch Anhang)

\bibitem{ICD-O SNOMED}
``Mapping of ICD-O Tuples to OncoTree Codes Using SNOMED CT Post-Coordination'' \newline
\url{https://pubmed.ncbi.nlm.nih.gov/35612082/}
% (eigene Vorgehensweise im Kontext „ICD-O vs. SNOMED CT vs OncoTree“: realisiert u.a. über ConceptMaps)

\bibitem{CrowdHEALTH}
``Interoperability Techniques in CrowdHEALTH project: The Terminology Service'' \newline
\url{https://www.ncbi.nlm.nih.gov/pmc/articles/PMC7085336/}
% (eine weitere Anwendung)

\bibitem{i2b2}
``i2b2 implemented over SMART-on-FHIR'' \newline
\url{https://www.ncbi.nlm.nih.gov/pmc/articles/PMC5961782/}
% (eine weitere Anwendung)

\bibitem{SNOMED-FHIR}
``SNOMED CT -- FHIR ConceptMap Translate'' \newline
\url{https://github.com/IHTSDO/snowstorm/blob/master/docs/fhir-resources/concept-map.md}
% (SNOMED CT verwendet natürlich selber ConceptMaps)

\bibitem{I-MAGIC}
``I-MAGIC: SNOMED CT to ICD-10-CM Map'' \newline
\url{https://www.nlm.nih.gov/research/umls/mapping_projects/snomedct_to_icd10cm.html}
% (I-MAGIC-Tool für ein differenziertes Mapping von Problems/SNOMED CT nach ICD-10)

\bibitem{MAGPIE}
``MAGPIE: Map Assisted Generation of Procedure and Intervention Encoding'' \newline
\url{https://magpie.nlm.nih.gov/demo}
% (MAGPIE-Tool für ein Mapping von Prozeduren zum amerikanischen ICD-9-CM bzw. ICD-10-PCS)

% Das Problem mit den obigen Beispielen (die technisch durchaus die FHIR ConceptMap-Ressource benutzen) ist, dass es um Mappings zwischen Codes von verschiedenen CodeSystemen geht, insb. unter Beteiligung von SNOMED CT. Es gibt wenig Publikationen und Informationen zum Mapping zwischen Versionen von Klassifikationen, d.h. von ICD-10-GM-2024 ó ICD-10-GM-2023 ó ICD-10-GM-2022 ó usw. (und das Ganze auch für OPS und ATC).

\bibitem{8-Steps}
``8 Steps to Success in ICD-10-CM/PCS Mapping: Best Practices to Establish Precise Mapping Between Old and New ICD Code Sets'' \newline
\url{https://library.ahima.org/doc?oid=106975} \newline
\url{https://pubmed.ncbi.nlm.nih.gov/22741510/}

\bibitem{ICD-8-9-10}
``Mapping three versions of the international classification of diseases to categories of chronic conditions '' \newline
\url{https://pubmed.ncbi.nlm.nih.gov/34007901/}

\bibitem{SNOMED Browser}
SNOMED CT Browser \newline
\url{https://browser.ihtsdotools.org/}
% Extraktion aller SNOMED CT-Konzepte (mit der Expression Contraint Lanugage (ECL)), die auf den gleichen ICD-10 – Code gemappt werden,
% z.B. mit der Anfrage: ^ [referencedComponentId, mapTarget] 447562003 |ICD-10 complex map refset| {{M mapGroup=#1, mapTarget="J45.9"}}
% => diese kann man interaktiv eingeben unter https://browser.ihtsdotools.org/ (oben rechts)

\end{thebibliography}
\endgroup
\end{comment}

\end{document}
