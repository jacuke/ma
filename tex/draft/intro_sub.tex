\begin{comment}
\bibitem{medinf-init}
Medizininformatik-Initiative \newline
\url{https://www.medizininformatik-initiative.de/de/start}
% Der Druck andererseits ist groß, mit vorhandenen Versorgungsdaten (insb. in der Medizininformatik-Initiative (LINK), in der auch wir aktiv sind) für die Forschung zu arbeiten.

\bibitem{fdpg}
Deutsches Forschungsdatenportal für Gesundheit \newline
\url{https://forschen-fuer-gesundheit.de}
% Hierzu gibt es eine Forschungsdaten für Gesundheits-Plattform (LINK), wo im großen Stil Daten aus sogenannten Datenintegrationszentren (DIZ) aus etwa 30 Universitätskliniken in Deutschland föderiert abgefragt werden. 
\end{comment}

Das Bundesinstitut für Arzneimittel und Medizinprodukte, kurz: BfArM, veröffentlicht jedes Jahr aktualisierte Versionen medizinischer Klassifikationen, wie zum Beispiel der Kodiersysteme ICD-10-GM und OPS. Die jährlichen Veröffentlichungen enthalten Überleitungstabellen von der neuen Version auf die jeweilige Vorgängerversion. Diese Änderungen in den Kodes erschweren die versionsübergreifende Analyse klinischer Daten, wie sie beispielsweise vom Deutsches Forschungsdatenportal für Gesundheit \cite{medinf-init} zur Verfügung gestellt werden. Ein technischer Ansatz das Code-Mapping-Problem zu lösen wäre die Verwendung von \emph{ConceptMaps} aus dem FIHR Standards für den Austausch elektronischer Gesundheitsdaten.

\section{Grundbegriffe}
\subsection{Klassifikationen ICD-10-GM und OPS}

Diese Arbeit bezieht sich auf die Systematischen Verzeichnisse von ICD-10-GM und OPS, das heißt der nach dem Kode strukturierten Veröffentlichungen der Kodiersysteme. 

\newpara{ICD-10-GM}

Zitiert aus \cite[Seite 97]{gaus2005dokumentation}:

\quoteBox{
Die International Statistical Classification of Diseases (ICD) geht auf das Jahr 1855 zurück
[...] und wird heute von der World Health Organisation (WHO) betreut. Die
10. Revision trat am 01.01.1993 in Kraft unter der Bezeichnung [...] ICD-10. Die deutschen
Ausgaben werden vom Deutschen Institut für Medizinische Dokumentation und Information (DIMDI)
als dem WHO Kooperationszentrum erarbeitet. Die am 01.01.2005 in Kraft
getretene Ausgabe heißt [...] ICD-10-GM 2005, wobei GM für German Modification steht.
}

Seit Mai 2020 ist das DIMDI in das BfArM eingegliedert. \cite{bfarmdimdi}

\newpage

Wiederum \cite[Seite 98ff]{gaus2005dokumentation}:

\quoteBox{
Die ICD-10-GM 2005 ist ein Ordnungssystem für \emph{Krankheiten (Diagnosen)}, das nach dem
\emph{Ordnungsprinzip Klassifikation} aufgebaut ist und insgesamt ca. 64 000 Klassen hat. Die
\emph{Notation} ist vier- oder fünfstellig. Sie besteht aus einem Buchstaben, einer zweistelligen
Zahl, einem Punkt (dient nur der Strukturierung, liefert keine Information und wird deshalb
bei der Stellenzahl nicht mitgezählt) und dann noch eine (vierstellige Notation) oder zwei
(fünfstellige Notation) einstellige Zahlen. [...] \\

Die ICD-10-GM 2005 ist hierarchisch geordnet. Bei der Notation bezeichnen die 3 Stellen
vor dem Punkt ein hierarchisches Niveau, das Raum für 2 500 gleichgeordnete Klassen bietet
und nur durch Überschriften weiter gegliedert ist. Die Stellen nach dem Punkt geben weitere
Niveaus an, d.h. die vierstellige Notation beschreibt 2, die fünfstellige Notation 3 hierarchische Niveaus. [...] \\

Ein allgemeines Problem der systematischen Anordnung von Diagnosen ist, ob eine Diagnose unter dem Ort ihrer Manifestation oder unter dem ihr zugrunde liegenden Krankheitsprozess eingeordnet werden soll. \emph{Beispiel:} Soll die Lungenentzündung unter der Lokalisation
Lunge oder unter dem Krankheitsprozess Entzündung eingeordnet werden? Werden alle
Krankheiten eines bestimmten Organs nebeneinander gestellt, so folgt die Systematik dem
topologisch-organspezifischen Aspekt (\emph{Topologie} = Lehre von der Lage und Anordnung der
Dinge im Raum). In einer Systematik können jedoch auch alle Krankheiten mit dem gleichen
Krankheitsprozess, z.B. alle Entzündungen, alle Autoimmunkrankheiten oder alle bösartigen
Neubildungen, nebeneinander gestellt werden. Diese Einteilung nennt man ätiologisch,
pathologisch oder nosologisch (\emph{Ätiologie} = die Krankheit auslösende Ursache, \emph{Pathologie
und Nosologie} = Lehre von den Krankheiten). Dieses Problem ist mit dem Ordnungsprinzip
Klassifikation nicht lösbar. Die [ICD-10-GM] benutzt im Wesentlichen den ätiologischen Aspekt.
}

Definition des Begriffs Klassifikation aus \cite[Seite 86f]{gaus2005dokumentation}:

\quoteBox{Von allen Ordnungsprinzipien ist die Klassifikation das einfachste. Es beruht auf dem
Grundsatz: "`Jedes Ding (jeder Sachverhalt) an seinen Platz"'. Das zu dokumentierende Sachgebiet wird in einzelne \emph{getrennte Sachverhalte} eingeteilt, die man als \emph{Klassen} bezeichnet. [...]
Die Klassen sind \emph{disjunkt}, d.h. sie schließen sich gegenseitig aus und überlappen sich nicht. Jede Klasse wird
durch einen Deskriptor repräsentiert. Die Klassen einer Klassifikation sind gleichzeitig
\emph{Äquivalenzklassen} von Begriffen. Im strengen Fall ist die Zuordnung einer Dokumentationseinheit zu einer Klasse eindeutig, d.h. eine Dokumentationseinheit wird genau einer
Klasse zugeteilt. Eine Klassifikation ist einfach und praktisch. Sie ist sozusagen das \emph{"`natürliche Ordnungsprinzip"'}. [...] \\

Ein \emph{Klassifikationssystem} -- ein Ordnungssystem, das nach dem Ordnungsprinzip Klassifikation aufgebaut ist -- muss vollständig sein. Vollständig sein bedeutet, dass die Klassen alle
Sachverhalte des dokumentarisch zu bearbeitenden Sachgebiets umfassen. [...] Durch Schaffung einer Klasse \emph{"`sonstiges"'} oder besser
durch die Schaffung mehrerer Klassen mit dem Zusatz „sonstiges“ wird
die geforderte Vollständigkeit des Klassifikationssystems [...]
formal erreicht. [...] Um auf die einzelnen Klassen bequem und sicher zugreifen zu können, werden sie \emph{hierarchisch} oder anderweitig \emph{systematisch} angeordnet.

%Um auf die einzelnen Klassen bequem und sicher zugreifen zu können, werden sie hierarchisch oder anderweitig systematisch angeordnet. [...]%Eine hierarchische oder anderweitig systematische Anordnung trägt, wie in den beiden folgenden Themen ausführlich behandelt wird, erheblich zur terminologischen Kontrolle bei. In der Klassifikation verbindet sich also das einfachste Ordnungsprinzip und die einfachste Form der terminologischen Kontrolle. Deshalb sind Klassifikationen weit verbreitet.
}

\newpage

Vom BfArM werden Notation und Begriff bezeichnet als Kode und Titel.

\newpara{OPS}

Ebenfalls zitiert aus \cite[Seite 101ff]{gaus2005dokumentation}:

\quoteBox{
In der Medizin werden nicht nur Krankheiten, sondern auch ärztliche Tätigkeiten dokumentarisch erfasst und – ebenso wie die Diagnosen für die Abrechnung und für wissenschaftliche
Zwecke verwendet. Die von der WHO erstmals 1978 herausgegebene International Classification of Procedures in Medicine (ICPM) wurde (ebenso wie die ICD-10) vom DIMDI ins
Deutsche übertragen und an deutsche Verhältnisse angepasst. Die am 01.01.2005 in Kraft
getretene Ausgabe heißt "`Operationen- und Prozedurenschlüssel -- Internationale Klassifikation der Prozeduren in der Medizin (OPS 2005)"'. Der OPS 2005 ist – wie die Bezeichnung \emph{"`Schlüssel"'} ausdrückt – eine Klassifikation mit numerischer Notation. Allerdings haben in einigen Bereichen die 10 gleichgeordneten Klassen nicht ausgereicht, deshalb treten
an der 5. und 6. Stelle der Notation gelegentlich Buchstaben auf. Von den insgesamt
14 000 Klassen betreffen etwa 70\% Operationen, ca. 15\% nichtoperative therapeutische
Maßnahmen, 8\% diagnostische Maßnahmen, 4\% bildgebende Diagnostik und 2\% ergänzende Maßnahmen. \\

Die \emph{Notation} des OPS 2005 ist vier- bis sechsstellig. Die erste Stelle enthält eine 1, 3, 5, 8
oder 9 und bezeichnet einen der eben genannten Bereiche (z.B. 1 = diagnostische Maßnahmen). Es folgt ein Bindestrich und eine dreistellige Zahl. Je nach Detaillierungsgrad ist damit
die Notation beendet oder es folgen noch ein Punkt und weitere 1 bis 2 Stellen. [...] In der fünften und sechsten Stelle der Notation (d.h. nach dem Punkt) können Ziffern durch
die Buchstaben x oder y ersetzt werden, es bedeutet x = sonstige und y = nicht näher bezeichnet.
}

\subsection{FHIR ConceptMap}

FHIR steht für "`Fast Healthcare Interoperability Resources"'. 

\newpara{HL7 FHIR}

Zitiert aus \cite[Seite 309]{fhir-heckmann}: 

\quoteBox{
"`Health Level 7"' wurde 1987 gegründet, um Standards für klinische Informationssysteme
zu erarbeiten. [...] % Der Name der Organisation nimmt Bezug auf die siebte Schicht (Application Layer) des ISO/OSI-Modells, und bringt damit zum Ausdruck, dass hier die Festlegung von anwendungsbezogenen Inhalten und Prozessen im Vordergrund steht, nicht die Spezifikation von z. B. Ubertragungsprotokollen (HLT International, 2021).
HL7 ist vom nationalen amerikanischen Normeninstitut (ANSI) seit 1994 akkreditiert.
HL7-verbundene Organisationen existieren inzwischen in über 40 Ländern. Die erste
nationale Partnergesellschaft wurde 1993 in Deutschland gegründet. Die Aufgabe dieser
sogenannten "`Affiliates"' ist es, die Verbreitung und Implementierung der internationalen
HL7-Standards in dem jeweiligen Land zu fördern und gegebenenfalls erforderliche Anpassungen und Erweiterungen für deren Nutzung zu entwickeln. FHIR ist die dritte Generation von Interoperabilitätsstandards aus der Feder von HL7. \\

Die Entwicklung begann im Jahre 2011 als Reaktion auf die Forderungen aus der Industrie nach einer standardisierten Lösung für die Entwicklung webbasierter Applikationen für das Gesundheitswesen. %Der erste Entwurf der FHIR-Spezifikation stammt aus der Feder von Grahame Grieve (AUS), Ewout Kramer (NL) und Lloyd McKenzie (CAD). Erste stabile Versionen von FHIR wurden 2014 und 2015 mit dem Zusatz ,,DSTU* (Draft Standard for Trial Use) publiziert und bereits von der Industrie zur Implementierung genutzt. Insbesondere die Version DSTU 2 wurde in den USA schnell aufgegriffen, obgleich der «»DSTU*-Zusatz. den nicht-normativen Status dieser Publikationen hervorhebt. 2017 wurde die dritte Version unter der Bezeichnung ,STU3" publiziert.
[...] Die vierte, 2018 veröffentlichte Version "`R4"' enthält erstmals normative Inhalte. % Die Entwicklung des Standards ist damit jedoch noch nicht abgeschlossen. Mit Hilfe eines ausgekliigelten Reifegradmodelles wird der Entwicklungsstand einzelner Teile der Spezifikation regelmiibig bewertet. Normativen Status erhiilt ein Artefakt dann, wenn es geméi8 der Abstimmungsregein bei HL7 ballotiert, umfassend getestet und von mindestens fiinf unabhiingigen Hersteller in mehr als einem Land unter produktiven Bedingungen implementiert wurde. Damit wird sicher- gestellt, dass nichts normativ (und damit als weitgehend unvertinderlich) deklariert wird, was nicht ausfiihrlich auf seine Nutzbarkeit unter Echtbedingungen hin getestet wurde.
}

\newpage

Ebenfalls aus \cite[Seite 310]{fhir-heckmann}:

\quoteBox{
Im Gegensatz zu älteren HL7-Standards [...] bedient sich FHIR erstmals moderner webbasierter Technologien, die auch jenseits des Gesundheitswesens weit verbreitet und bei Entwicklern bekannt und beliebt sind. FHIR legt den Fokus auf die query-getriebene Art der
Kommunikation, die dazu dient, Anwendungen und Anwendern die benötigten Daten zum
benötigten Zeitpunkt in adäquatem Umfang und Format zur Verfügung zu stellen. Dazu
bedient sich FHIR dem Representational State Transfer (kurz: "`REST"'), einem Paradigma,
das die Struktur und das Verhalten des Word Wide Webs abstrahiert und basierend auf denselben Technologien und Grundlagen eine einheitliche Form der Kommunikation von verteilten Systemen und Webservices definiert. Die auszutauschenden Datenobjekte werden
als "`Ressourcen"' bezeichnet, auf denen jeweils sogenannte "`CRUD"'-Interaktionen (create, read, update, delete) über die HTTP-Methoden GET, PUT, POST und DELETE ausgeführt werden können.
%FHIR definiert auf dieser Grundlage etwa 150 verschiedene, im Gesundheitswesen reevante, Objekttypen (im REST-Kontext als ,.Ressourcentypen" bezeichnet) und deren Bezichungen untereinander (im REST-Kontext als ,.Links* bezeichnet) (Tab. 1),
%‘Zur Serialisierung der Ressourcen fir die Speicherung und Ubertragung erlaubt FHIR sowohl das JSON- als auch das XML-Format. Server-Applikationen sind angehalten,
}

Die Autorin hebt außerdem folgende Eigenschaften hervor:

\begin{itemize}
\item "`FHIR als Baukasten"': Standards und Leitfäden in FHIR sind selbst Ressourcen. Sie können erweitert und für verschiedene Zwecke eingesetzt werden. 
\item "`FHIR als Community-Standard"': Die FHIR-Spezifikation ist frei verfügbar und wird gemeinschaftlich entwickelt, statt von einem geschlossenen Gremium. 
\end{itemize}

Die am meisten verbreitetsten Implementationen von FHIR sind der als freie Software lizenzierte API-Server \emph{HAPI}, sowie der proprietäre Terminologie-Server \emph{Ontoserver}, welcher unter anderem im australischen Gesundheitssystem eingesetzt wird. %Für mehr dazu siehe 
\cite{braunstein2022health}

\begin{comment}
«HL7\textsuperscript{\textregistered} FHIR\textsuperscript{\textregistered} (im Folgenden "`FHIR"' genannt) ist der modernste Interoperabilitäts-Stan-\newline dard aus der Produktfamilie von Health Level 7 International (kurz: "`HL7"'), einer internationalen Standardisierungsorganisation für das Gesundheitswesen, die in der Vergangenheit schon viele erfolgreiche und weit verbreitet genutzte Standards, wie zum Beispiel HL7 Version 2 oder HL7 CDA (Clinical Document Architecture) hervorgebracht hat. [...] HL7 wurde 1987 gegründet, um Standards für klinische Informationssysteme zu erarbeiten. [...] FHIR ist die dritte Generation von Interoperabilitätsstandards aus der Feder von HL7. Die Entwicklung begann im Jahre 2011 als Reaktion auf die Forderungen aus der Industrie nach einer standardisierten Lösung für die Entwicklung webbasierter Applikationen für das Gesundheitswesen.» \cite{fhir-heckmann}

"`Health Level 7 wurde 1987 gegründet, um Standards für klinische Informationssysteme zu erarbeiten. [...] FHIR ist die dritte Generation von Interoperabilitätsstandards aus der Feder von HLR7. Die Entwicklung begann im Jahre 2011 als Reaktion auf die Forderungen aus der Industrie nach einer standardistieren Lösung für die Entwicklung webbasierter Applikationen für das Gesundheitswesen."' \cite[Seite 309]{fhir-heckmann}



\begin{comment}
2 Einfiihrung in HL7* FHIR®

«Health Level 7* wurde 1987 gegriindet, um Standards fir klinische Informationssysteme
zu erarbeiten, Der Name der Organisation nimmt Bezug auf die siebte Schicht (Applica-
tion Layer) des ISO/OSI-Modells, und bringt damit zum Ausdruck, dass hier die Fest-
legung von anwendungsbezogenen Inhalten und Prozessen im Vordergrund steht, nicht die
Spezifikation von z. B. Ubertragungsprotokollen (HLT International, 2021).

HL7 ist vom nationalen amerikanischen Normeninstitut (ANSI) seit 1994 akkreditiert.
HLT verbundene, Organisationen, existieren inzwischen in tiber 40 Laindern. Die erste
nationale Partnergesellschaft wurde 1993 in Deutschland gegriindet. Die Aufgabe dieser
sogenannten ,,Affiliates* ist es, die Verbreitung und Implementierung der internationalen
HL7-Standards in dem jeweiligen Land zu férdern und gegebenenfalls erforderliche An-
passungen und Erweiterungen fiir deren Nutzung zu entwickeln, FHIR ist die dritte Gene-
ration von Interoperabilititsstandards aus der Feder von HLT.

Die Entwicklung began im Jahre 2011 als Reaktion auf die Forderungen aus der Indus-
trie nach einer isierten Losung fiir die Entwi ikationen fiir
das Gesundheitswesen. Der erste Entwurf der FHIR-Spezifikation stammt aus der Feder
von Grahame Grieve (AUS), Ewout Kramer (NL) und Lloyd McKenzie (CAD). Erste sta-
bile Versionen von FHIR wurden 2014 und 2015 mit dem Zusatz ,,DSTU* (Draft Standard
for Trial Use) publiziert und bereits von der Industrie zur Implementierung genutzt. Ins-
besondere die Version DSTU 2 wurde in den USA schnell aufgegriffen, obgleich der
«»DSTU*-Zusatz. den nicht-normativen Status dieser Publikationen hervorhebt. 2017 wurde
die dritte Version unter der Bezeichnung ,STU3" publiziert. Die vierte, 2018 veréffent-
lichte Version ,,R4* enthiilt erstmals normative Inhalte, Die Entwicklung des Standards ist
damit jedoch noch nicht abgeschlossen. Mit Hilfe eines ausgekliigelten Reifegradmodelles
wird der Entwicklungsstand einzelner Teile der Spezifikation regelmiibig bewertet. Norma-
tiven Status erhiilt ein Artefakt dann, wenn es geméi8 der Abstimmungsregein bei HL7
ballotiert, umfassend getestet und von mindestens fiinf unabhiingigen Hersteller in mehr
als einem Land unter produktiven Bedingungen implementiert wurde. Damit wird sicher-
gestellt, dass nichts normativ (und damit als weitgehend unvertinderlich) deklariert wird,
was nicht ausfiihrlich auf seine Nutzbarkeit unter Echtbedingungen hin getestet wurde.

 

 

 

2.1 FHIR als webbasierter Standard

> — Eine webbasierte Anwendung (kurz: Web App") funktioniert nach dem Client-
Server-Modell. Im Gegensatz zu klassischen Desktopanwendungen werden Web
Apps meist nicht lokal installiert, sondern tiber einen Browser ausgefiihrt und sind
damit unabhiingig vom Betriebssystem (Thin Client). Die Datenverarbeitung fin-







310 . Heckmann

 

~ abgeschen von Datenvalidierungen zur Uberpriifung von Benutzereingaben ~
vorzugsweise auf einem entfernten Webserver statt. Die Kommunikation zwischen
Client und Server erfolgt tiber das HTTP-Protokoll

Im Gegensatz zu ilteren HL7-Standards wie HL7 Version 2, einem ereignis-orientierten,
nachrichtenbasierten Austauschformat, oder HL7 Version 3, einer abstrakten, modell-
orientierten Spezifikation, bedient sich FHIR erstmals moderner webbasierter Techno-
logien, die auch jenseits des Gesundheitswesens weit verbreitet und bei Entwicklern be-
kannt und beliebt sind, FHIR legt den Fokus auf die query-getriebene Art der
Kommunikation, die dazu dient, Anwendungen und Anwendern die bendtigten Daten zum
bendtigten Zeitpunkt in adiquatem Umfang und Format zur Verfigung zu stellen, Daz
bedient sich FHIR dem Representational State Transfer (kurz: ,REST“), einem Paradigma,
das die Struktur und das Verhalten des Word Wide Webs abstrahiert und basierend auf den-
selben Technologien und Grundlagen eine einheitliche Form der Kommunikation von ver-
teilten Systemen und Webservices definiert. Die auszutauschenden Datenobjekte werden
als ,Ressourcen bezeichnet, auf denen jeweils sogenannte CRUD“Interaktionen
(create, read, update, delete) iiber die HTTP-Methoden GET, PUT, POST und DELETE
ausgefithrt werden kénnen

FHIR definiert auf dieser Grundlage etwa 150 verschiedene, im Gesundheitswesen re-
evante, Objekttypen (im REST-Kontext als ,.Ressourcentypen" bezeichnet) und deren Be-
zichungen untereinander (im REST-Kontext als ,.Links* bezeichnet) (Tab. 1),

‘Zur Serialisierung der Ressourcen fir die Speicherung und Ubertragung erlaubt FHIR
sowohl das JSON- als auch das XML-Format. Server-Applikationen sind angehalten,

 

 

   

 

 

Tab. 1 Beispicle fiir in FHIR definierte Ressourcentypen Quelle: HL7 International: FHIR Spezi-
fikation Version 4.0.1.: Ressourcentypen. http:/hI7.org/fhir/resourcelist. html
‘Name des

 

 

 

 

Bereich
Patient ‘Name, Geburisdatum, Adresse, Gesel ‘Administration
Encounter Datum und Dauer des Aufenthaltes, aang it ‘Administration
(stationiir/ambulant), verantwortliche Organisation, Ort
(Station, Zimmer, Bettplatz)
Condition Diagnose-Code (z. B. ICD-10), Diagnosedatum, Klinik

  

Schweregrad, Klinischer Status (gesichert/Verdacht)

Observation Durchgefihrte Messung (zB. Klinisehe Chemie, Fieber, Diagnostik
Puls, Blutdruck), Messergebnis, verwendetes Geri,
Zeitpunkt der Messung, Normbereich

 

Chargtiem erbrache Leising, Unfan, ZipunkVZsivaum, Anahi, | Abrechoung
s, Zu-/Abschkig
‘AuditEvent zion = Duco, ‘Art und Umfang des Zugriffes, | Sicherheit

 

ire-Komponenten bzw. Benutzer
ResearchSubject Nevtnupfong ces die,
Zaire ee etinchos,zugewicocerStudicaaen, ets

Einwilligungen

    

Forschung


\end{comment}

\newpara{REST}
\label{rest-paradigm}

Representational State Transfer ist --wie oben erwähnt-- ein Paradigma für die Softwarearchitektur von verteilten System, welches erstmals in \cite{rest-fielding} dargelegt wurde. Es besteht aus sechs Bedingungen: 

\begin{enumerate}
\item Kommunikation per Client-Server-Modell.
\item Zustandslosigkeit: Jede Nachricht zwischen Client und Server muss alle Information enthalten, die zum Verständnis der Nachricht benötigt wird. 
\item Erhöhung der Effizienz durch Einsatz von Caching.
\item Einheitliche Schnittstellen: 
\begin{enumerate}
\item Jede Ressource wird durch eine URL identifiziert.
\item Ressourcen werden verändert durch Repräsentationen, zum Beispiel strukturierte Dateiformate.
\item Die Inhalte der Nachrichten sind selbstbeschreibend.
\item Informationen werden vom Anwendungsserver dynamisch bereitgestellt über Hypermedien wie HTTP.
\end{enumerate}
\item Mehrschichtigkeit: Komplexität wird hinter den offenen Endpunkten verborgen. 
\item "`Code on Demand"': Es soll die Möglichkeit für Clients bestehen, Verfahren zur Verwendung neuer Medientypen vom Server abfragen zu können.
\end{enumerate}

\vspace{-0.4em}

\newpara{ConceptMap}
\label{conceptmap-structure}

Die ConceptMap ist eine FHIR Ressource, die zum Mapping zwischen Kodiersystemen vorgesehen ist. Die Definition ist frei verfügbar, zum Beispiel für FHIR Release 4 unter \cite[FHIR R4]{conceptmap-r4}. Die Komponenten, die für ein Mapping zwischen Versionen der ICD-10-GM und des OPS mindestens benötigt werden, sind folgende:

\begin{comment}
\begin{customIndentRight2}
%\renewcommand{\arraystretch}{1.2}
\setlength{\tabcolsep}{12pt}
\begin{tabular}{ccccc}
%\multicolumn{4}{c}{Bestandteil} \\
%\hline
ConceptMap & & & & \\
\drawHookArrow & url & & & \\
\drawHookArrow & id & & & \\
\drawHookArrow & group & & & \\
               & \drawHookArrow & source & & \\
               & \drawHookArrow & target & & \\
               & \drawHookArrow & element & & \\
               &                & \drawHookArrow & code & \\
               &                & \drawHookArrow & target & \\
               &                &                & \drawHookArrow & code \\
               &                &                & \drawHookArrow & equivalence \\
%               & \drawHookArrow & unmapped & & \\
%               &                & \drawHookArrow & mode & \\
\end{tabular}
\end{customIndentRight2}
\end{comment}

\begin{figure}[H]
\centering
%\renewcommand{\arraystretch}{1.2}
\normalsize
\setlength{\tabcolsep}{12pt}
\begin{tabular}{ccccc}
%\multicolumn{4}{c}{Bestandteil} \\
%\hline
ConceptMap & & & & \\
\drawHookArrow & url & & & \\
\drawHookArrow & id & & & \\
\drawHookArrow & group & & & \\
               & \drawHookArrow & source & & \\
               & \drawHookArrow & target & & \\
               & \drawHookArrow & element & & \\
               &                & \drawHookArrow & code & \\
               &                & \drawHookArrow & target & \\
               &                &                & \drawHookArrow & code \\
               &                &                & \drawHookArrow & equivalence \\
%               & \drawHookArrow & unmapped & & \\
%               &                & \drawHookArrow & mode & \\
\end{tabular}
\caption{Hierarchischer Aufbau der FHIR Ressource "`ConceptMap"'.}
\end{figure} 

\vspace{-1em}

\begin{itemize}
%\item ConceptMap: A statement of relationships from one set of concepts to one or more other concepts - either concepts in code systems, or data element/data element concepts, or classes in class models. \newline A map from one set of concepts to one or more other concepts
\item \emph{url}: Eine absolute URI, \emph{Uniform Resource Identifier}, welche einmal pro ConceptMap enthalten ist und diese global eindeutig identifiziert. Idealerweise handelt es sich um eine existierende URL. Laut Spezifikation ist die Angabe eigentlich optional, aber sie wird von HAPI vorausgesetzt. %An absolute URI that is used to identify this concept map when it is referenced in a specification, model, design or an instance; also called its canonical identifier. This SHOULD be globally unique and SHOULD be a literal address at which at which an authoritative instance of this concept map is (or will be) published. This URL can be the target of a canonical reference. It SHALL remain the same when the concept map is stored on different servers. \newline Canonical identifier for this concept map, represented as a URI (globally unique) \newline § wird von HAPI benötigt
\item \emph{id}: Die logische ID der Ressource; das heißt sie ist innerhalb eines Servers eindeutig. Sie kann nicht verändert werden und ist wie die URL pro ConceptMap einmalig eingetragen, sowie optional. Aber es ist natürlich sinnvoll die ID dafür zu verwenden, Zweck und Umfang der ConceptMap zu kennzeichnen. % The logical id of the resource, as used in the URL for the resource. Once assigned, this value never changes. \newline Logical id of this artifact § kann nicht verändert werden
\item \emph{group}: Eine Gruppe an Mappings, die alle das selbe Start- und Zielsystem haben. Eine ConceptMap kann mehrere Gruppen enthalten. %A group of mappings that all have the same source and target system. \newline Same source and target systems
\item \emph{source}: Eine absolute URI, die das Start- beziehungsweise Ursprungssystem der Konzepte für das Mapping identifiziert. Sie kann einmal pro \emph{group} angegeben werden. Bei einer in sich geschlossenen ConceptMap, das heißt ohne Referenz auf eine FHIR "`Code System"' Ressource, kann dieser Eintrag auch frei gewählt sein. %An absolute URI that identifies the source system where the concepts to be mapped are defined. \newline 	Source system where concepts to be mapped are defined
\item \emph{target}: Wie \emph{source} nur für das Zielsystem. %An absolute URI that identifies the target system that the concepts will be mapped to. \newline Target system that the concepts are to be mapped to
\item \emph{element}: Mappings für ein einzelnes Konzept der Ursprungsversion auf ein oder mehrere Konzepte der Zielversion. Pro \emph{group} sind mehrere \emph{element}-Einträge möglich. %Mappings for an individual concept in the source to one or more concepts in the target. \newline Mappings for a concept from the source set
\item \emph{code}: Ein Eintrag pro \emph{element}, der dieses identifiziert. %Identity (code or path) or the element/item being mapped. \newline Identifies element being mapped
\item \emph{target}: Konzept(e) der Zielversion, für das Mapping von \emph{element}. %A concept from the target value set that this concept maps to.\newline Concept in target system for element
\item \emph{code}: Kode, der das Zielkonzept identifiziert. %Identity (code or path) or the element/item that the map refers to.\newline Code that identifies the target element
\item \emph{equivalence}: Die Relation zwischen Konzepten der Ursprungs- und Zielversion. \newline In FHIR R5 heißt diese Komponente \emph{relationship}. %The equivalence between the source and target concepts (counting for the dependencies and products). The equivalence is read from target to source (e.g. the target is 'wider' than the source). \newline § Auflisten, alle Versionen
%\item \emph{unmapped}: What to do when there is no mapping for the source concept. "Unmapped" does not include codes that are unmatched, and the unmapped element is ignored in a code is specified to have equivalence = unmatched. \newline What to do when there is no mapping for the source concept \newline § provided
\end{itemize}

Wie die Relationen angegeben werden, unterscheidet sich zwischen FHIR Release 4 und 5 \cite{conceptmap-r5}. Release 6 wird hierbei unverändert zu R5 bleiben. Für die Überleitungen zwischen den Versionen der ICD-10-GM und des OPS gibt es vier Fälle: 

\begin{enumerate}
\item Es gibt keine Veränderung. R4 und R5: \emph{equivalent}.
\item Der Kode wurde verändert. R4: \emph{relatedto}. R5: \emph{related-to}.
\item Die Überleitung erfolgt auf mehrere mögliche Kodes. R4: \emph{wider}. \newline R5: \emph{source-is-narrower-than-target}.
\item Der Kode wurde entfernt. R4: \emph{unmatched}. \newline In R5 ist kein \emph{target}-Eintrag mehr notwendig, sondern es wird stattdessen auf der gleichen Ebene wie der \emph{code} eine Komponente \emph{noMap} mit Wert \emph{true} eingetragen. 
\end{enumerate}

Zusätzlich gibt es noch die Komponente \emph{unmapped} pro \emph{group}. Hier kann angegeben werden, wie Konzepte behandelt werden sollen, die zwar in dem unter \emph{source} referenzierten Kodiersystem enthalten sind, für die es allerdings keinen Eintrag im Mapping als \emph{element$\rightarrow$code} gibt. Da bei den Überleitungen zwischen den Versionen der ICD-10-GM und des OPS meistens die große Mehrheit der Kodes unverändert bleibt, wäre es also sinnvoll, um die Größe der ConceptMap gering zu halten, alle Kodes mit \emph{equivalent}-Relation gar nicht einzutragen und stattdessen den \emph{unmapped}-Modus "`provided"' zu verwenden, der diese Kodes einfach wie angegeben als "`Ziel"' des Mappings zurückgibt, was ebenfalls der Äquivalenz entspricht. Leider wird aber \emph{unmapped} in der aktuellen HAPI-Version nicht unterstützt und von Ontoserver nur im beschränkten Maße: \cite{fhir-unmapped}.

\section{Verwandte Arbeiten}

In \cite{hund} wird die Verwendung der ICD-10-GM und des OPS im deutschen Gesundheitssystem diskutiert, sowie der Nutzung die daraus resultierenden Daten in der Forschung. Als eine der Herausforderungen wurde dabei der Vergleich über die verschiedenen, jährlich veröffentlichten Versionen identifiziert. Um diese zu lösen wurde eine Software-Library entwickelt: \cite{medicats}. Darin ist das Ermitteln von Überleitungen ausgehend von jeweils einem Kode in rückwärts chronologischer Reihenfolge implementiert. Allerdings war das letzte Update von 2020 und das Hinzufügen neuer Versionen erfordert die Programmierung mehrerer Klassen pro Version. Außerdem setzt das Kompilieren der Library das Vorhandensein von Dateien in einer Größe von circa 1 GB voraus und das Laden der Daten pro Applikationsstart ist ein relativ langer Prozess. 

\newpage

Im US-amerikanisches Gesundheitssystem erhält die ICD-10-CM, Clinical Modification, ebenfalls jedes Jahr eine neue Version. Zum Beispiel in \cite{manchikanti2015tragedy} werden die jährlichen Mehrkosten des Einsatzes im Vergleich zu ICD-9-CM betont. Programmatische Ansätze zur Überleitung zwischen den Versionen innerhalb der ICD-10-CM sind allerdings nicht bekannt. Es gibt jedoch Mappings zwischen ICD-9-CM und ICD-10-CM, wie zum Beispiel in \cite{info:doi/10.2196/14325}.

%\subsection{Mappings mit Verwendung von FHIR}
\subsection{FHIR Mappings}

Seit Veröffentlichung der FHIR Release 4 in 2019 werden bevorzugt ConceptMap Ressourcen für Mappings im klinischen Kontext verwendet. Interessant ist beispielsweise die Verwendung mehrerer \emph{group}-Einträge in \cite{saripalle2019representing}, um UMLS in FHIR ConceptMaps abzubilden. Das \emph{Unified Medical Language System} ist ein Ansatz, mittels eines Kompendiums kontrollierter Vokabulare zwischen medizinischen Terminologien zu übersetzen. Einen ähnlichen Zweck verfolgt das \emph{Observational Medical Outcomes Partnership} (OMOP) \emph{Common Data Model} (CDM) der \emph{Observational Health Data Sciences and Informatics} (OHDSI) Initiative. Ein Beispiel für die Verwendung dessen in einer ConceptMap ist \cite{10.1093/jamiaopen/ooae045}. Eine weitere Vorgehensweise wie in \cite{fhir-snomed} ist wiederum ein Mapping über SNOMED CT, wobei \emph{Systematized Nomenclature of Medicine Clinical Terms} die umfassendste medizinische Terminologie ist.

\begin{comment}

--------------

UMLS

\bibitem{UMLS FHIR}
``An Interoperable UMLS Terminology Service Using FHIR'' \newline
\url{https://www.mdpi.com/1999-5903/12/11/199}
% (Vorgehensweise im Kontext UMLS: das hatten wir ja in der Master-Vorlesung)

\bibitem{UMLS FHIR 2}
``Representing UMLS knowledge using FHIR Terminological Resources'' \newline
\url{https://ieeexplore.ieee.org/abstract/document/8983305}
% (wie vorher zu UMLS; siehe auch Anhang)

--------------

\bibitem{MENDS-on-FHIR}
``MENDS-on-FHIR: Leveraging the OMOP common data model and FHIR standards for national chronic disease surveillance'' \newline
\url{https://www.medrxiv.org/content/10.1101/2023.08.09.23293900v2}
% (wie vorher zum Kontext OHDSI; siehe auch Anhang)

\end{comment}


\subsection{SNOMED CT als Interlingua}

Basierend auf dem letzten Ansatz wäre also eine Möglichkeit das Mapping zwischen Versionen der ICD-10-GM und des OPS über SNOMED CT als Interlingua zu gestalten. Es wurden für ICD-10-GM $\leftrightarrow$ SNOMED CT Mappings in beide Richtungen bereits elektronische Ressourcen entwickelt. Zum Beispiel gibt es I-MAGIC  von der \emph{National Library of Medicine} \cite{i-magic} um differenzierte Mappings von SNOMED CT nach ICD-10-CM interaktiv zu erstellen. SNOMED International stellt einen Browser \cite{snomed-browser} zur Verfügung, der mit Expression Contraint Lanugage Queries für ICD-10 Kodes die Abfrage der mehreren zutreffenden SNOMED CT Konzepte ermöglicht. Das Mapping Tool \cite{snomed-map} funktioniert je nach Eingabe in beide Richtungen. 

Beispielsweise werden für den ICD-10-CM M21.4, "`Flat foot [pes planus] (acquired)"' folgende SNOMED CT IDs zurückgegeben:

%\begin{centernss}
{
\renewcommand{\arraystretch}{1.2}
%\setlength{\tabcolsep}{12pt}
\begin{tabular}{ll}
%\uzlemph{Kode} & \uzlemph{Titel} \\
SCTID & Begriff \\
\hline
203533003 & Peroneal spastic flat foot \\
90374001 & Acquired spastic flat foot \\
53226007 & Talipes planus \\
44480001 & Acquired flexible flat foot \\
203531001 & Hypermobile flat foot \\
203534009 & Acquired pes planus \\
203532008 & Rigid flat foot \\
\end{tabular}
}
%\end{centernss}

Für die andere Richtung existiert eine 1:1 Kardinalität. \cite{WHO-FIC}

Ein Ansatz die für ein Mapping am besten geeignetste SCTID zu ermitteln wird in \cite{icd10-to-snomed} vorgestellt. SNOMED CT Konzepte sind hierarchisch in |is-a|-Relationen angeordnet und so wird hier das "`passendste"' Konzept dem  ranghöchsten gleichgesetzt. Im Beispiel oben ist 90374001--"`Acquired spastic flat foot"' ein Kind von 53226007--"`Talipes planus"'. Letztere und 203534009--"`Acquired pes planus"' sind allerdings gleich ranghoch. 

Eine andere Idee wäre das Mapping anhand der Ähnlichkeit von Begriff in SNOMED CT und Titel in ICD-10 zu berechnen. Dazu gibt es Algorithmen wie \cite{jaro-winkler}, welcher für das M21.4--"`Flat foot [pes planus] (acquired)"'--Beispiel "`Acquired pes planus"' als beste Übereinstimmung ermittelt und \cite{smith-waterman-gotoh} "`Talipes planus"'.

Es wäre also möglich geeignete Konzepte für ein Mapping vorzuschlagen, aber die endgültige Auswahl müsste medizinisches Fachpersonal treffen. In \cite{manual-map} wurde ein Mapping der schwedischen ICD-10 Version auf SNOMED CT manuell durchgeführt. Die Urteilerübereinstimmung der beiden Kodierer betrug 89\%, was als moderates Ergebnis bewertet wird.

Ein weiteres Problem ist, dass der Vergleich der deutschen Titel dadurch erschwert wird, dass die Übersetzung der deutschen SNOMED CT Version noch nicht weit fortgeschritten ist; siehe \cite{gmds}. Ähnlich ernüchternd sind Ergebnisse von Mappings zwischen OPS und SNOMED CT wie in \cite{schulz2019aligning}.

In dieser Arbeit wird also ein Mapping zwischen den verschiedenen Versionen innerhalb der ICD-10-GM und dem OPS  auf Basis der vom BfArM veröffentlichten Überleitungstabellen vorgenommen. Sobald ein operativer Reifegrad der Mappings zwischen diesen Kodiersystemen und SNOMED CT erreicht ist, wäre es wahrscheinlich immer noch hilfreich die Versionen innerhalb von ICD-10-GM und OPS auf zum Beispiel die aktuellste Version anzugleichen. 


\begin{comment}
M21.4
Flat foot [pes planus] (acquired)

Jaro-Winkler: Acquired pes planus 
Smith Waterman Gotoh: Talipes planus
\end{comment}


%\subsection{SNOMED}


\begin{comment}

\bibitem{ICD-O SNOMED}
``Mapping of ICD-O Tuples to OncoTree Codes Using SNOMED CT Post-Coordination'' \newline
\url{https://pubmed.ncbi.nlm.nih.gov/35612082/}
% (eigene Vorgehensweise im Kontext „ICD-O vs. SNOMED CT vs OncoTree“: realisiert u.a. über ConceptMaps)

^ Ohlsen/Ingenerf



\bibitem{SNOMED Browser}
SNOMED CT Browser \newline
\url{https://browser.ihtsdotools.org/}
% Extraktion aller SNOMED CT-Konzepte (mit der Expression Contraint Lanugage (ECL)), die auf den gleichen ICD-10 – Code gemappt werden,
% z.B. mit der Anfrage: ^ [referencedComponentId, mapTarget] 447562003 |ICD-10 complex map refset| {{M mapGroup=#1, mapTarget="J45.9"}}
% => diese kann man interaktiv eingeben unter https://browser.ihtsdotools.org/ (oben rechts)

^ Mapping ICD-10 -> SNOMED CT


\bibitem{I-MAGIC}
``I-MAGIC: SNOMED CT to ICD-10-CM Map'' \newline
\url{https://www.nlm.nih.gov/research/umls/mapping_projects/snomedct_to_icd10cm.html}
% (I-MAGIC-Tool für ein differenziertes Mapping von Problems/SNOMED CT nach ICD-10)

^ Mapping SNOMED CT -> ICD-10




\bibitem{ICD-10 to SNOMED CT}
``Evaluation of an Automated Mapping from ICD-10 to SNOMED CT'' \newline
icd10-to-snomed
\url{https://american-cse.org/csci2022-ieee/pdfs/CSCI2022-2lPzsUSRQukMlxf8K2x89I/202800b612/202800b612.pdf}
% ODER: Man mappt die Klassen (unterhalb der Klassenebene) auf präzisere SNOMED CT-Codes (das sind mehrere, d.h. welchen nimmt man?) und garantiert so jahres- bzw. versionsübergreifende Auswertungen?

^ Problem beim automatisierten Mapping



\bibitem{SNOMED-FHIR}
``SNOMED CT -- FHIR ConceptMap Translate'' \newline
\url{https://github.com/IHTSDO/snowstorm/blob/master/docs/fhir-resources/concept-map.md}
% (SNOMED CT verwendet natürlich selber ConceptMaps)
=> IHTSDO / Snowstorm

??????????????? 






\bibitem{OHDSI FHIR}
``Building Interoperable FHIR-Based Vocabulary Mapping Services: A Case Study of OHDSI Vocabularies and Mappings'' \newline
\url{https://ebooks.iospress.nl/doi/10.3233/978-1-61499-830-3-1327}
% (Vorgehensweise im Kontext OHDSI, einem langjährigen Real-World-Evidence-Projekt, wo heterogene Daten auf ein Standard-Datenmodell (OMOP) inkl. Standard-Terminologie (Athena-Tool) gemappt wird.) 

=> Referenz

\bibitem{OHDSI Athena}
OHDSI Athena \newline
\url{https://cdn.who.int/media/docs/default-source/classification/who-fic-network/whofic_terminology_mapping_guide.pdf?sfvrsn=2cae387c_7&download=true}
% Ein ganz anderer Ansatz wäre, die Klassifikations-Codes auf Terminologie-Codes (insb. SNOMED CT) zu mappen, um nochmal ganz anders vorzugehen. Ein ähnliches Vorgehen gibt es im OHDSI-Projekt, wo ein interessantes Terminologie-Tool „Athena“ (LINK) verwendet wird.





% Das Problem mit den obigen Beispielen (die technisch durchaus die FHIR ConceptMap-Ressource benutzen) ist, dass es um Mappings zwischen Codes von verschiedenen CodeSystemen geht, insb. unter Beteiligung von SNOMED CT. Es gibt wenig Publikationen und Informationen zum Mapping zwischen Versionen von Klassifikationen, d.h. von ICD-10-GM-2024 ó ICD-10-GM-2023 ó ICD-10-GM-2022 ó usw. (und das Ganze auch für OPS und ATC).

\end{comment}


%%%%%%%%%%%%%%%%%%%%%%%%%%%%%%%%%%%%%%%%%%%%%%%%%%%%%%%%%%%%%


\begin{comment}
\begingroup
\renewcommand{\section}[2]{}
\begin{thebibliography}{99}

-------------

+-----+
| IGN |
+-----+

\bibitem{ICD-10-GM}
ICD-10-GM Kodiersystem \newline
\url{https://www.bfarm.de/DE/Kodiersysteme/Klassifikationen/ICD/ICD-10-GM/_node.html}

\bibitem{CrowdHEALTH}
``Interoperability Techniques in CrowdHEALTH project: The Terminology Service'' \newline
\url{https://www.ncbi.nlm.nih.gov/pmc/articles/PMC7085336/}
% (eine weitere Anwendung)

\bibitem{i2b2}
``i2b2 implemented over SMART-on-FHIR'' \newline
\url{https://www.ncbi.nlm.nih.gov/pmc/articles/PMC5961782/}
% (eine weitere Anwendung)

+----+
| NO |
+----+

\bibitem{8-Steps}
``8 Steps to Success in ICD-10-CM/PCS Mapping: Best Practices to Establish Precise Mapping Between Old and New ICD Code Sets'' \newline
\url{https://library.ahima.org/doc?oid=106975} \newline
\url{https://pubmed.ncbi.nlm.nih.gov/22741510/}

\bibitem{ICD-8-9-10}
``Mapping three versions of the international classification of diseases to categories of chronic conditions '' \newline
\url{https://pubmed.ncbi.nlm.nih.gov/34007901/}

\bibitem{fhir-concept}
FHIR ConceptMap2 \newline
\url{https://hl7.org/fhir/conceptmap.html}

\bibitem{fhir-mule}
Mulesoft FHIR R4 ConceptMap Library \newline
\url{https://anypoint.mulesoft.com/exchange/org.mule.examples/fhir-r4-conceptmap-library/}

\bibitem{fhir-term}
FHIR Terminology Module \newline
\url{https://build.fhir.org/terminology-module.html}
%https://www.youtube.com/watch?v=Hd8A_gBnpzk

\bibitem{MAGPIE}
``MAGPIE: Map Assisted Generation of Procedure and Intervention Encoding'' \newline
\url{https://magpie.nlm.nih.gov/demo}
% (MAGPIE-Tool für ein Mapping von Prozeduren zum amerikanischen ICD-9-CM bzw. ICD-10-PCS)

\end{thebibliography}
\endgroup
\end{comment}



\begin{comment}
\section{Quali}

WHO-FIC Classifications and Terminology Mapping

Principles

\begin{enumerate}
\item Establish use case(s) before developing the map
\item Clearly define the purpose, scope, and directionality of the map.
\item Maps should be unidirectional and single purposed. Separate maps should be
maintained for bidirectional maps (to support both a forward and a backward
map table). Such unidirectional maps can be handy to support data continuity for
epidemiological and longitudinal studies. Maps should not be reversed.
\item Develop clear and transparent documentation that is freely available to all and
describes the purpose, scope, limitations, and methodology of the map.
\item Ideally, the producers of both terminologies in any map participate in the
mapping effort to ensure that the result accurately reflects their terminologies'
meaning and usage. At a minimum, both terminology producers should define
the primary purpose and parameters of the mapping task, review and verify the
map, develop the plan for testing and validation, and devise a cost-effective
strategy for building, maintaining, and enhancing the map over time.
\item Map developers should agree on team members' competencies, knowledge, and
skills at the project's onset. Ideally, users of the map also participate in its design
and testing to ensure that it fits its intended purpose.
\item Quality Assurance (QA) and Usage Validation: QA and usage validation is
ensuring the reproducibility, traceability, usability and comparability of the maps.
Establish the QA and usage validation protocols at the beginning of the project
and apply them throughout the mapping process. Factors that may be involved
in quality assurance include quality-assurance rules, testing (test protocols, pilot
testing), and quality metrics (such as computational metrics or precisely defined
cardinality, equivalence, and conditionality). Usage validation of maps is an
independent process involving users of the maps (not developer of the maps) in
order to determine whether the maps are fit for purpose (e.g. do the end users
reach to the correct code in the target terminology when using manual and
automated maps etc.). Usage validation is essential to ensure the integrity of the
information from source data to the final coding. Key principles for usage
validation of maps include:
\begin{enumerate}
\item use of the ground truth of the original source data1 (e.g. diagnosis as
written in the medical record) as the reference point;
\item compare the original source data with the end results of the following two
processes
\newline i. Coding of original source data with a source terminology – map
code(s) of source terminology to code(s) of target terminology
\newline ii. Coding of original source data with target terminology
\item statistically significant sample size that is representative of the target
terminology and its prototypical use case settings.
\item Usage validation of automated maps should always include human (i.e.
manual) validation
\end{enumerate}
Clear documentation of the QA process and validation procedures is essential
in this step in the mapping process.
If conducting a pilot test is feasible, it will improve the QA/validation process.
Mapping is an iterative process that will improve overtime as it is used in real
settings.

\item Dissemination: Upon publication and release, include information about release
mechanisms, release cycle, versioning, source/target information, licence
agreement requirements, and a feedback mechanism for users. Dissemination of
maps should also include documentation, as stated above, describing the
purpose, scope, limitations, and methodology used to create the maps.
\item Maintenance: establish an ongoing maintenance mechanism, release cycle, types
and drivers of changes, and versioning of maps. The maintenance phase should
include an outline of the overall lifecycle plan for the map, open transparent
resolution mechanism for mapping problems, continuous improvement process,
and decision process around when an update is required. Whenever maps are
updated, the cycle of QA and validation must be repeated.
\item When conducting mapping manually, it is recommended to provide map
specialists with the necessary tools and documentation to drive consistency when
building the map. These include such items as the tooling environment (workflow
details and resources related to both source and target schemes); source and
target browsers, if available; technical specifications (use case, scope, definitions);
editorial mapping principles or rules to ensure consistency of the maps,
particularly where human judgement is required; and implementation guidance.
Additionally, it is best practice to provide an environment that supports dual
independent authoring of maps as this is thought to reduce bias between human
map specialists. Developing a consensus management process to aid in resolving
discrepancies and complex issues is also beneficial.
\item In computational mapping, it is advisable to include resources to ensure
consistency when building a map using a computational approach, including a
description of the tooling environment, when human intervention would occur,
documentation (e.g. the rules used in computerized algorithms), and
implementation guidance. It is also advisable to always compute the accuracy
and error rate of the maps. It is also essential to manually verify and validate the
computer-generated mapping lists. Such manual checking is necessary for the
quality assurance process, as maps generated automatically will almost always
contain errors. Such manually verified maps can also help train the machine-
learning model when maps for different sections of terminologies are being
generated sequentially.
\item Cardinality is a metric in mapping that must be clearly defined regarding what is
being linked between source and target and how the cardinalities are counted.
For example, SNOMED CT codes for functional impairments are semantically
different from ICF codes. A 1:1 map between the two does not mean semantic
equivalence. In terms of counting, what SNOMED International considers to be a
1:1 map includes what others may consider being a 1:many map.
\item Level of equivalence, such as broader, narrower, or overlap, should be specified.
\item Maps must be machine-readable to optimize their utility.
\item ICD-11: When creating maps using ICD-11 and other WHO systems, mapping
into the Foundation Component comes first, then maps to MMS could created
through linearization aggregation
\end{enumerate}

ISO/TR 12300

Prinzipien:

\begin{enumerate}
\item Each map should have a (preferably single) declared purpose
\item Scenarios are developed and articulated to define the requirements for the map table
\item The map table should be in a machine processable format
\item Identify each version of each terminological resource as a version of the Map Table
\item Members of the project team should have knowledge of both of the terminological
resource and experience in their practical application
\item Establish the extent to which the conventions and rules of each terminological resource
will be followed
\item Custodians of terminological resources should be involved in mapping projects
\item The automated and manual methods applied should be transparent and documented
\item Every map should describe the direction of the map
\item Cardinality of each individual map should be clearly specified
\item Any loss or gain of meaning should be made explicit and risk assessed \newline
All maps should demonstrate the degree of equivalence
\item All mapping projects should make explicit the guidelines and heuristics applied in
developing and interpreting the maps when implemented
\item Documentation supporting the map should describe the map data structures, distribution
format and licensing arrangements
\item Every mapping project should have a quality assurance plan which includes testing and
validation
\item Every mapping project should have a consensus management process
\item Maps should be maintained and routinely updated during their lifespan
\item Every map should have a maintenance and evaluation plan, which includes the
mechanisms for version control
\item Maps should have continuing improvement processes
\item Every map should have supporting documentation to assist implementation and use
\item Map development and maintenance is best managed through a team
\end{enumerate}

ISO 21564

Determinants of map quality

\begin{enumerate}
\item Common categorical structure (ja)
\item Shared semantic domain (ja)
\item Language and Translation (keine Übersetzung)
\item Equivalence Identification / Publication (ja)
\item Equivalence Assessment (abweichungen zw. source und target)
\item Map Set Outliers
\item Clear documentation of the purpose of the map
\item Currency of the map (zeitliche Nähe)
\item Business Arrangements
\item Methodology Documentation
\item Percentage of map validated
\item Method of validation
\item Decision making (klare Prozesse)
\item Tools used to develop or maintain the map (ja)
\item Workforce
\item Governance (Entscheidungen)
\item Map Maintenance
\end{enumerate}
\end{comment}

\subsection{Qualitätsstandards für Mappings}
\label{quali-map}

Es folgt eine Zusammenfassung verschiedener Grundprinzipien und Maßstäbe für das Erstellen von Mappings im Gesundheitswesen. In eckigen Klammern sind die Listennummern in den jeweiligen Quellen referenziert:

\textbf{\emph{A}} "`Classifications and Terminology Mapping -- Principles and Best Practice"', World Health Organization, \emph{Arbeitsgruppe:} Family of International Classifications. \cite{WHO-FIC}

\textbf{\emph{B}} "`Principles of Mapping between Terminological Systems"', International Organization for Standardization, \emph{Komitee:} Health Informatics. \cite{ISO12300}

\textbf{\emph{C}} "`Terminology Resource Map Quality Measures"', International Organization for Standardization, \emph{Komitee:} Health Informatics. \cite{ISO21564}

\vspace{1em}

\begin{enumerate}
\item Einsatzszenarien müssen vor Entwicklung des Mappings etabliert sein. Zweck, Umfang und die Richtung müssen klar definiert sein. Mappings sollten jeweils eine Richtung und einen Zweck haben. \emph{[A1, A2, A3, B1, B2, B9]}
\item Die Urheber des Kodiersystem sollten in der Erstellung und Validierung des Mappings beteiligt sein. Mit dem Kodiersystemen verbundenen Konventionen und Richtlinien sollten eingehalten werden. \emph{[A5, B6, B7]}
\item Das Team sollte nach den vorhandenen Kompetenzen strukturiert sein. Es ist förderlich, wenn ein Interesse externer Organisationen besteht. \emph{[A6, B5, B20, C9, C15]}
\item Dokumentationen über Zweck, Umfang, Einschränkungen und Implementierung des Mappings müssen erstellt werden. Bei automatisierten, beziehungsweise maschinell erstellten Mappings müssen Informationen über die verwendeten Technologien und Algorithmen bereitgestellt werden. Es ist wichtig die Ergebnisse manuell zu überprüfen. \emph{[A4, A11, B8, B12, B19, C7, C10, C14]}
\item Qualitätssicherung und Validierung sind essentiell; Testprotokollen müssen existieren und zukünftige Anwender des Mappings sollten involviert sein. \emph{[A7, B4, B14, B15, C11, C12]}
\item Die Veröffentlichung des Mappings muss Informationen über Versionen, Update-Zyklen und Lizenzen enthalten. Ein Wartungsprozess muss definiert sein. Die Weiterentwicklung sollte durch Feedback von Anwendern gewährleistet sein und über einen Konsensbildung erfolgen. Mappings sollten möglichst aktuell sein. \emph{[A8, A9, B13, B16, B17, B18, C8, C13, C16, C17]}
\item Kardinalität, Relationen, sowie Ursprung und Ziel des Mappings müssen klar definiert sein. \emph{[A12, A13, B10, B11, C4, C5, C6]}
\item Mappings sollten maschinenlesbar sein. \emph{[A14, B3]}
\end{enumerate}

Für diese Arbeit nicht relevant sind Punkte zu manuell erstellten Mappings \emph{[A10]}, spezifischen Kodiersystemen wie ICD-11 \emph{[A15]}, Eigenschaften von Ursprung- und Zielsystem, falls diese abweichen \emph{[C1, C2]}, Übersetzung natürlicher Sprachen \emph{[C3]}. 

\section{Aufbau \& Beiträge dieser Arbeit}

Die Arbeit ist in drei Teile untergliedert:

\begin{enumerate}
\item Ein Integrationsprozess für die BfArM-Daten, der für ICD-10-GM und OPS möglichst gleich funktioniert und die Aufnahme einer neuen Version möglichst einfach macht.
\item Zwei Suchalgorithmen, um Überleitungen der Kodes versionsübergreifend und chronologisch in beide Richtungen zu bestimmen. 
\item Eine Web-Applikation, welche die Ergebnisse der vorherigen zwei Punkte anzeigt und FHIR ConceptMaps für das versionsübergreifende Mapping von ICD-10-GM und OPS generiert. 
\end{enumerate}
