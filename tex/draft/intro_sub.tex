Medizininformatik-Initiative
\texttt{https://www.medizininformatik-initiative.de/de/start}

Deutsches Forschungsdatenportal für Gesundheit
\texttt{https://forschen-fuer-gesundheit.de}

\section{Grundbegriffe}
\subsection{Klassifikationen ICD-10-GM und OPS}
\subsection{FHIR ConceptMap}

«HL7\textsuperscript{\textregistered} FHIR\textsuperscript{\textregistered} (im Folgenden "`FHIR"' genannt) ist der modernste Interoperabilitäts-Stan-\newline dard aus der Produktfamilie von Health Level 7 International (kurz: "`HL7"'), einer internationalen Standardisierungsorganisation für das Gesundheitswesen, die in der Vergangenheit schon viele erfolgreiche und weit verbreitet genutzte Standards, wie zum Beispiel HL7 Version 2 oder HL7 CDA (Clinical Document Architecture) hervorgebracht hat. [...] HL7 wurde 1987 gegründet, um Standards für klinische Informationssysteme zu erarbeiten. [...] FHIR ist die dritte Generation von Interoperabilitätsstandards aus der Feder von HL7. Die Entwicklung begann im Jahre 2011 als Reaktion auf die Forderungen aus der Industrie nach einer standardisierten Lösung für die Entwicklung webbasierter Applikationen für das Gesundheitswesen.» \cite{fhir-heckmann}

"`Health Level 7 wurde 1987 gegründet, um Standards für klinische Informationssysteme zu erarbeiten. [...] FHIR ist die dritte Generation von Interoperabilitätsstandards aus der Feder von HLR7. Die Entwicklung begann im Jahre 2011 als Reaktion auf die Forderungen aus der Industrie nach einer standardistieren Lösung für die Entwicklung webbasierter Applikationen für das Gesundheitswesen."' \cite[Seite 309]{fhir-heckmann}

FHIR \cite{braunstein2022health}

\newpara{REST}

REST \cite{rest-fielding}

\newpara{ConceptMap}

http://hl7.org/fhir/R4/conceptmap.html

\begin{customIndentRight2}
%\renewcommand{\arraystretch}{1.2}
\setlength{\tabcolsep}{12pt}
\begin{tabular}{ccccc}
%\multicolumn{4}{c}{Bestandteil} \\
%\hline
ConceptMap & & & & \\
\drawHookArrow & url & & & \\
\drawHookArrow & id & & & \\
\drawHookArrow & group & & & \\
               & \drawHookArrow & source & & \\
               & \drawHookArrow & target & & \\
               & \drawHookArrow & element & & \\
               &                & \drawHookArrow & code & \\
               &                & \drawHookArrow & target & \\
               &                &                & \drawHookArrow & code \\
               &                &                & \drawHookArrow & equivalence \\
%               & \drawHookArrow & unmapped & & \\
%               &                & \drawHookArrow & mode & \\
\end{tabular}
\end{customIndentRight2}

\begin{itemize}
\item ConceptMap: A statement of relationships from one set of concepts to one or more other concepts - either concepts in code systems, or data element/data element concepts, or classes in class models. \newline A map from one set of concepts to one or more other concepts
\item url: An absolute URI that is used to identify this concept map when it is referenced in a specification, model, design or an instance; also called its canonical identifier. This SHOULD be globally unique and SHOULD be a literal address at which at which an authoritative instance of this concept map is (or will be) published. This URL can be the target of a canonical reference. It SHALL remain the same when the concept map is stored on different servers. \newline Canonical identifier for this concept map, represented as a URI (globally unique) \newline § wird von HAPI benötigt
\item id: The logical id of the resource, as used in the URL for the resource. Once assigned, this value never changes. \newline Logical id of this artifact § kann nicht verändert werden
\item group:  A group of mappings that all have the same source and target system. \newline Same source and target systems
\item source: An absolute URI that identifies the source system where the concepts to be mapped are defined. \newline 	Source system where concepts to be mapped are defined
\item target: An absolute URI that identifies the target system that the concepts will be mapped to. \newline Target system that the concepts are to be mapped to
\item element: Mappings for an individual concept in the source to one or more concepts in the target. \newline Mappings for a concept from the source set
\item code: Identity (code or path) or the element/item being mapped. \newline Identifies element being mapped
\item target: A concept from the target value set that this concept maps to.\newline Concept in target system for element
\item code: Identity (code or path) or the element/item that the map refers to.\newline Code that identifies the target element
\item equivalence: The equivalence between the source and target concepts (counting for the dependencies and products). The equivalence is read from target to source (e.g. the target is 'wider' than the source). \newline § Auflisten, alle Versionen
\item unmapped: What to do when there is no mapping for the source concept. "Unmapped" does not include codes that are unmatched, and the unmapped element is ignored in a code is specified to have equivalence = unmatched. \newline What to do when there is no mapping for the source concept \newline § provided
\end{itemize}

\section{Verwandte Arbeiten}
\subsection{Normen für Mappings}

Health Informatics, Terminology resource map quality measures (MapQual) \cite{ISO21564}

%Terminoloy vs Classification (terminology easier to re-use; but class. map to current version)

Health informatics -- Principles of mapping between terminological resources \cite{ISO12300}

\subsection{Medicats}

\begin{comment}
\bibitem{Medicats}
Medicats Library \newline
\url{https://github.com/hhund/medicats}

\bibitem{Medicats Thesis}
``Medical Classification and Terminology Systems in a Secondary Use Context: Challenges and Perils'' \newline
\url{https://www.researchgate.net/publication/320962066_Medical_Classification_and_Terminology_Systems_in_a_Secondary_Use_Context_Challenges_and_Perils}
\end{comment}

\subsection{ConceptMap für UMLS}

using multiple groups \cite{saripalle2019representing}

\begin{comment}
\bibitem{UMLS FHIR}
``An Interoperable UMLS Terminology Service Using FHIR'' \newline
\url{https://www.mdpi.com/1999-5903/12/11/199}
% (Vorgehensweise im Kontext UMLS: das hatten wir ja in der Master-Vorlesung)

\bibitem{UMLS FHIR 2}
``Representing UMLS knowledge using FHIR Terminological Resources'' \newline
\url{https://ieeexplore.ieee.org/abstract/document/8983305}
% (wie vorher zu UMLS; siehe auch Anhang)
\end{comment}

\subsection{Interlingua SNOMED CT}

SNOMED \cite{icd10-to-snomed}

\section{Aufbau \& Beiträge dieser Arbeit}







%\subsection{SNOMED}
%
%WHO-FIC Classifications and Terminology Mapping -- Principles and Best Practice 


\begin{comment}
\bibitem{WHO-FIC-Class}
``WHO-FIC Classifications and Terminology Mapping -- Principles and Best Practice'' \newline
\url{https://cdn.who.int/media/docs/default-source/classification/who-fic-network/whofic_terminology_mapping_guide.pdf?sfvrsn=2cae387c_7&download=true}

\bibitem{SNOMED Browser}
SNOMED CT Browser \newline
\url{https://browser.ihtsdotools.org/}
% Extraktion aller SNOMED CT-Konzepte (mit der Expression Contraint Lanugage (ECL)), die auf den gleichen ICD-10 – Code gemappt werden,
% z.B. mit der Anfrage: ^ [referencedComponentId, mapTarget] 447562003 |ICD-10 complex map refset| {{M mapGroup=#1, mapTarget="J45.9"}}
% => diese kann man interaktiv eingeben unter https://browser.ihtsdotools.org/ (oben rechts)

\bibitem{CrowdHEALTH}
``Interoperability Techniques in CrowdHEALTH project: The Terminology Service'' \newline
\url{https://www.ncbi.nlm.nih.gov/pmc/articles/PMC7085336/}
% (eine weitere Anwendung)

\bibitem{ICD-10 to SNOMED CT}
``Evaluation of an Automated Mapping from ICD-10 to SNOMED CT'' \newline
\url{https://american-cse.org/csci2022-ieee/pdfs/CSCI2022-2lPzsUSRQukMlxf8K2x89I/202800b612/202800b612.pdf}
% ODER: Man mappt die Klassen (unterhalb der Klassenebene) auf präzisere SNOMED CT-Codes (das sind mehrere, d.h. welchen nimmt man?) und garantiert so jahres- bzw. versionsübergreifende Auswertungen?

\bibitem{OHDSI FHIR}
``Building Interoperable FHIR-Based Vocabulary Mapping Services: A Case Study of OHDSI Vocabularies and Mappings'' \newline
\url{https://ebooks.iospress.nl/doi/10.3233/978-1-61499-830-3-1327}
% (Vorgehensweise im Kontext OHDSI, einem langjährigen Real-World-Evidence-Projekt, wo heterogene Daten auf ein Standard-Datenmodell (OMOP) inkl. Standard-Terminologie (Athena-Tool) gemappt wird.) 

\bibitem{OHDSI Athena}
OHDSI Athena \newline
\url{https://cdn.who.int/media/docs/default-source/classification/who-fic-network/whofic_terminology_mapping_guide.pdf?sfvrsn=2cae387c_7&download=true}
% Ein ganz anderer Ansatz wäre, die Klassifikations-Codes auf Terminologie-Codes (insb. SNOMED CT) zu mappen, um nochmal ganz anders vorzugehen. Ein ähnliches Vorgehen gibt es im OHDSI-Projekt, wo ein interessantes Terminologie-Tool „Athena“ (LINK) verwendet wird.

\bibitem{SNOMED-FHIR}
``SNOMED CT -- FHIR ConceptMap Translate'' \newline
\url{https://github.com/IHTSDO/snowstorm/blob/master/docs/fhir-resources/concept-map.md}
% (SNOMED CT verwendet natürlich selber ConceptMaps)

=> IHTSDO / Snowstorm

\bibitem{I-MAGIC}
``I-MAGIC: SNOMED CT to ICD-10-CM Map'' \newline
\url{https://www.nlm.nih.gov/research/umls/mapping_projects/snomedct_to_icd10cm.html}
% (I-MAGIC-Tool für ein differenziertes Mapping von Problems/SNOMED CT nach ICD-10)

\bibitem{MAGPIE}
``MAGPIE: Map Assisted Generation of Procedure and Intervention Encoding'' \newline
\url{https://magpie.nlm.nih.gov/demo}
% (MAGPIE-Tool für ein Mapping von Prozeduren zum amerikanischen ICD-9-CM bzw. ICD-10-PCS)

% Das Problem mit den obigen Beispielen (die technisch durchaus die FHIR ConceptMap-Ressource benutzen) ist, dass es um Mappings zwischen Codes von verschiedenen CodeSystemen geht, insb. unter Beteiligung von SNOMED CT. Es gibt wenig Publikationen und Informationen zum Mapping zwischen Versionen von Klassifikationen, d.h. von ICD-10-GM-2024 ó ICD-10-GM-2023 ó ICD-10-GM-2022 ó usw. (und das Ganze auch für OPS und ATC).

\end{comment}


%%%%%%%%%%%%%%%%%%%%%%%%%%%%%%%%%%%%%%%%%%%%%%%%%%%%%%%%%%%%%


\begin{comment}
\begingroup
\renewcommand{\section}[2]{}
\begin{thebibliography}{99}

\bibitem{fhir-term}
FHIR Terminology Module \newline
\url{https://build.fhir.org/terminology-module.html}
%https://www.youtube.com/watch?v=Hd8A_gBnpzk

\bibitem{fhir-concept}
FHIR ConceptMap2 \newline
\url{https://hl7.org/fhir/conceptmap.html}

\bibitem{fhir-mule}
Mulesoft FHIR R4 ConceptMap Library \newline
\url{https://anypoint.mulesoft.com/exchange/org.mule.examples/fhir-r4-conceptmap-library/}

\bibitem{medinf-init}
Medizininformatik-Initiative \newline
\url{https://www.medizininformatik-initiative.de/de/start}
% Der Druck andererseits ist groß, mit vorhandenen Versorgungsdaten (insb. in der Medizininformatik-Initiative (LINK), in der auch wir aktiv sind) für die Forschung zu arbeiten.

\bibitem{fdpg}
Deutsches Forschungsdatenportal für Gesundheit \newline
\url{https://forschen-fuer-gesundheit.de}
% Hierzu gibt es eine Forschungsdaten für Gesundheits-Plattform (LINK), wo im großen Stil Daten aus sogenannten Datenintegrationszentren (DIZ) aus etwa 30 Universitätskliniken in Deutschland föderiert abgefragt werden. 

\bibitem{ICD-10-GM}
ICD-10-GM Kodiersystem \newline
\url{https://www.bfarm.de/DE/Kodiersysteme/Klassifikationen/ICD/ICD-10-GM/_node.html}

-------------

\bibitem{MENDS-on-FHIR}
``MENDS-on-FHIR: Leveraging the OMOP common data model and FHIR standards for national chronic disease surveillance'' \newline
\url{https://www.medrxiv.org/content/10.1101/2023.08.09.23293900v2}
% (wie vorher zum Kontext OHDSI; siehe auch Anhang)

\bibitem{ICD-O SNOMED}
``Mapping of ICD-O Tuples to OncoTree Codes Using SNOMED CT Post-Coordination'' \newline
\url{https://pubmed.ncbi.nlm.nih.gov/35612082/}
% (eigene Vorgehensweise im Kontext „ICD-O vs. SNOMED CT vs OncoTree“: realisiert u.a. über ConceptMaps)

\bibitem{i2b2}
``i2b2 implemented over SMART-on-FHIR'' \newline
\url{https://www.ncbi.nlm.nih.gov/pmc/articles/PMC5961782/}
% (eine weitere Anwendung)

\bibitem{8-Steps}
``8 Steps to Success in ICD-10-CM/PCS Mapping: Best Practices to Establish Precise Mapping Between Old and New ICD Code Sets'' \newline
\url{https://library.ahima.org/doc?oid=106975} \newline
\url{https://pubmed.ncbi.nlm.nih.gov/22741510/}


\bibitem{ICD-8-9-10}
``Mapping three versions of the international classification of diseases to categories of chronic conditions '' \newline
\url{https://pubmed.ncbi.nlm.nih.gov/34007901/}



\end{thebibliography}
\endgroup
\end{comment}




\begin{comment}

\section{Grundlagen}

Zitiert

\begin{centernss}
\setlength{\fboxsep}{.03\linewidth}\color{black!20}\fbox{
\normalcolor\begin{minipage}[t]{.8\linewidth}
7.2
Grundsätzliches zum Ordnungsprinzip Klassifikation
Von allen Ordnungsprinzipien ist die Klassifikation das einfachste. Es beruht auf dem
Grundsatz: „Jedes Ding (jeder Sachverhalt) an seinen Platz“. Das zu dokumentierende Sach-
gebiet wird in einzelne getrennte Sachverhalte eingeteilt, die man als Klassen bezeichnet.
Bildlich gesprochen werden also die einzelnen Sachverhalte eines Sachgebiets in die Fächer
oder Schubladen eines Schrankes eingeordnet. Die einzelnen Fächer oder Klassen sind dis-
junkt, d.h. sie schließen sich gegenseitig aus und überlappen sich nicht. Jede Klasse wird
durch einen Deskriptor repräsentiert. Die Klassen einer Klassifikation sind gleichzeitig
Äquivalenzklassen von Begriffen. Im strengen Fall ist die Zuordnung einer Dokumentati-
onseinheit zu einer Klasse eindeutig, d.h. eine Dokumentationseinheit wird genau einer
Klasse zugeteilt. Eine Klassifikation ist einfach und praktisch. Sie ist sozusagen das „natür-
liche Ordnungsprinzip“.
Beispiele für Klassifikationen sind:
xMutters Wäscheschrank,
xdie meisten Magazinordnungen und Ersatzteillager (falls sie nicht als reine Lager nach
Signatur, Ersatzteilnummer oder dergleichen geordnet sind),
xdas Aufstellungsprinzip in einer Freihandbibliothek.Ordnungsprinzip Klassifikation

Aufteilung des Sachgebiets

Ein Klassifikationssystem -- ein Ordnungssystem, das nach dem Ordnungsprinzip Klassifi-
kation aufgebaut ist -- muss vollständig sein. Vollständig sein bedeutet, dass die Klassen alle
Sachverhalte des dokumentarisch zu bearbeitenden Sachgebiets umfassen. Ein Klassifika-
tionssystem kann auch mit einem Mosaik verglichen werden. Dabei entsprechen die Klassen
den Mosaiksteinchen, das komplette Ordnungssystem dem vollständigen Mosaikbild und das
Ordnungsprinzip Klassifikation der Kunsttechnik des Mosaiks. Wichtig ist, dass in dem Mo-
saikbild keine Steinchen fehlen bzw. dass alle Sachverhalte im Ordnungssystem vorhanden
sind und es keine Dokumentationseinheiten und keine Suchfragen gibt, die nicht in die Klas-
sifikation eingeordnet werden können. Durch Schaffung einer Klasse „sonstiges“ oder besser
durch die Schaffung mehrerer Klassen mit dem Zusatz „sonstiges“ (z.B. sonstige Knochen-
erkrankungen, sonstige Blutkrankheiten, sonstige Krankheiten des Verdauungstraktes) wird
die geforderte Vollständigkeit des Klassifikationssystems sozusagen durch ein Hintertürchen
formal erreicht. Andererseits gehört es zum Prinzip der Mosaiktechnik, dass an jeder Stelle
des Mosaiks nur ein Steinchen sein kann, d.h. dass die Klassen disjunkt sind.
Die Klassen einer Klassifikation können unterschiedlich große Sachverhalte abdecken. Im
Zentrum der bearbeiteten Thematik sind die Klassen meist sehr speziell und eng, am Rande
der bearbeiteten Thematik dagegen allgemein und weit. Auch in dieser Hinsicht passt der
Vergleich mit einem Mosaik: An wichtigen Stellen will der Künstler detailliert darstellen
und verwendet kleine Steinchen, an anderen Stellen mit größeren Steinen nur grob
skizzieren.
\end{minipage}
}
\normalcolor
\end{centernss}

\begin{customIndentRight}
Von allen Ordnungsprinzipien ist die Klassifikation das einfachste. Es beruht auf dem
Grundsatz: „Jedes Ding (jeder Sachverhalt) an seinen Platz“. Das zu dokumentierende Sach-
gebiet wird in einzelne getrennte Sachverhalte eingeteilt, die man als Klassen bezeichnet.
Bildlich gesprochen werden also die einzelnen Sachverhalte eines Sachgebiets in die Fächer
oder Schubladen eines Schrankes eingeordnet. Die einzelnen Fächer oder Klassen sind dis-
junkt, d.h. sie schließen sich gegenseitig aus und überlappen sich nicht. Jede Klasse wird
durch einen Deskriptor repräsentiert. Die Klassen einer Klassifikation sind gleichzeitig
Äquivalenzklassen von Begriffen. Im strengen Fall ist die Zuordnung einer Dokumentati-
onseinheit zu einer Klasse eindeutig, d.h. eine Dokumentationseinheit wird genau einer
Klasse zugeteilt. Eine Klassifikation ist einfach und praktisch. Sie ist sozusagen das „natür-
liche Ordnungsprinzip“.
\end{customIndentRight}
\end{comment}


