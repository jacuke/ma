% % % % % % % %
% Packages
% % % % % % % %

%?
\usepackage{tabularx}
\usepackage{longtable}
\usepackage{parskip}
\setlength{\parindent}{0pt}
\setlength{\mathindent}{0pt}
%/?

\usepackage{pgfplots}

\usepackage{multirow}
\usepackage{fancyvrb}
\usepackage{tikz,lipsum,lmodern}
\usepackage[most]{tcolorbox}
\usepackage{float}
\usepackage[german]{struktex}

%\usepackage{arydshln}

\pgfplotsset{compat=1.18}

\makeatletter
\def\verbatim@font{\linespread{0.9}\normalfont\ttfamily}
\makeatother

% % % % % % % %
% Environments
% % % % % % % %

\newenvironment{customIndentRight}{

\setlength{\leftskip}{1.5cm}
\setlength{\rightskip}{1.5cm}}{

\setlength{\leftskip}{0cm}
\setlength{\rightskip}{0cm}}

\newenvironment{customIndentRight2}{

\setlength{\leftskip}{0.65cm}
\setlength{\rightskip}{0.65cm}}{

\setlength{\leftskip}{0cm}
\setlength{\rightskip}{0cm}}

% % % % % % % %
% Commands
% % % % % % % %

\newcommand{\bfarmer}{
\textbf{\texttt{bfarmer}}
}

\newcommand{\newpara}[1]{
\vspace{0.7em}{\bfseries #1}
}

\newcommand{\heightNS}{8}

\newcommand{\struktogrammA}[3]{
\begin{center}
\small
\begin{struktogramm}(149,#1)[#2]
#3
\end{struktogramm}
\end{center}
}

\newcommand{\struktogrammAO}[2]{
\begin{center}
\small
\begin{struktogramm}(149,#1)
#2
\end{struktogramm}
\end{center}
}

\newcommand{\struktkommentar}[1]{
\begin{customIndentRight2}
%\small
#1
\end{customIndentRight2}
%\begin{centernss}
%\small
%#1
%\end{centernss}
}

\newcommand{\umsteigerTabelleInnenWeite}{
.57\textwidth
}

\newcommand{\drawHookArrow}{
\begin{tikzpicture}
   \draw[->] (0,0) |- (16pt,-8pt);
   \path[draw=white] (0,-8pt) -- (-16pt,-11pt);
\end{tikzpicture}
}

\newcommand{\umsteigerTabelleCodeBreak}{
\\[-2pt]
}
\newcommand{\umsteigerTabelleCodeBreakEnd}{
\\[-10pt]
}
\newcommand{\umsteigerTabelleSonstiges}[1]{
\begin{minipage}[t]{\umsteigerTabelleInnenWeite}%\vspace{-7pt}
%\begin{itemize}[leftmargin=*]
#1
%\end{itemize}
\end{minipage}
}

\begin{comment}
\newcommand{\umsteigerTabelleAnmerkung}[1]{
\multicolumn{3}{ |l| }{\emph{Anmerkung:}
\begin{minipage}[t]{.75\textwidth}
#1 %\vspace{3pt}
\end{minipage}
} \\
}
\end{comment}

\newcommand{\umsteigerTabelleZeile}[3]{
#1 & #2 & #3 \\
\hline
}

\newcommand{\umsteigerTabelleKopf}{
\textbf{Version} & \multicolumn{2}{ c| }{\textbf{Abweichungen zwischen den Versionen}} \\
}

\newcommand{\umsteigerTabelleZeileU}[2]{
#1 & URL & \texttt{#2}  \\ \cline{2-3}
}

\newcommand{\umsteigerTabelleZeileUCU}[4]{
\multirow{2}{*}{#1} & URL & \texttt{#2}  \\ \cline{2-3}
& Kodes & \begin{minipage}[t]{\umsteigerTabelleInnenWeite} \texttt{#3} \end{minipage} \\ \cline{2-3}
& Umsteiger & \begin{minipage}[t]{\umsteigerTabelleInnenWeite} \texttt{#4} \end{minipage} \\ \cline{2-3}
}

\newcommand{\umsteigerTabelleZeileUU}[3]{
\multirow{2}{*}{#1} & URL & \texttt{#2}  \\ \cline{2-3}
& Umsteiger & \begin{minipage}[t]{\umsteigerTabelleInnenWeite} \texttt{#3} \end{minipage} \\ \cline{2-3}
}

\newcommand{\umsteigerTabelleZeileUCUS}[5]{
\multirow{2}{*}{#1} & URL & \texttt{#2}  \\ \cline{2-3}
& Kodes & \begin{minipage}[t]{\umsteigerTabelleInnenWeite} \texttt{#3} \end{minipage} \\ \cline{2-3}
& Umsteiger & \begin{minipage}[t]{\umsteigerTabelleInnenWeite} \texttt{#4} \end{minipage} \\ \cline{2-3}
& Sonstiges & \umsteigerTabelleSonstiges{#5}  \\ \cline{2-3}
}

\newcommand{\umsteigerTabelleZeileUS}[3]{
\multirow{2}{*}{#1} & URL & \texttt{#2}  \\ \cline{2-3}
& Sonstiges & \umsteigerTabelleSonstiges{#3}  \\ \cline{2-3}
}

\newcommand{\umsteigerTabelleZeileUV}[3]{
\multirow{2}{*}{#1} & URL & \texttt{#2}  \\ \cline{2-3}
& Verzeichnis & \texttt{#3}  \\ \cline{2-3}
}

\newcommand{\umsteigerTabelleZeileLetzte}[3]{
\multirow{2}{*}{#1} & Kodes & \begin{minipage}[t]{\umsteigerTabelleInnenWeite} \texttt{#2} \end{minipage} \\ \cline{2-3}
& Sonstiges & \umsteigerTabelleSonstiges{#3}  \\ \cline{2-3}
}

\newcommand{\codeBox}[2]{
\begin{tcolorbox}[center,width=#1\linewidth,
    colback=white,colframe=black,boxrule=.5pt]
\texttt{\small #2}
\end{tcolorbox}
}

\newcommand{\codeBoxL}[1]{
\begin{tcolorbox}[width=.365\linewidth,
    colback=white,colframe=black,boxrule=.5pt]
\texttt{\small #1}
\end{tcolorbox}
\vspace{-4pt}
}

\newcommand{\codeBoxLD}[2]{
\begin{tcolorbox}[width=.365\linewidth,
    colback=white,colframe=black,boxrule=.5pt]
\texttt{\small #1}
\tcblower
\texttt{\small #2}
\end{tcolorbox}
\vspace{-4pt}
}

\newcommand{\codeBoxDouble}[3]{
\begin{tcolorbox}[center,width=#1\textwidth,
    colback=white,colframe=black,boxrule=.5pt]
\texttt{\small #2}
\tcblower
\texttt{\small #3}
\end{tcolorbox}
}

